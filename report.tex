\documentclass[fleqn]{article}
\usepackage{cmap}
\usepackage{booktabs} 				% toprule/midrule/bottomrule для таблиц
\usepackage{multirow}
\usepackage[left=1in, right=1in, top=1in, bottom=1in]{geometry}
\usepackage{mathexam}
\usepackage{mathtext} 				% русские буквы в фомулах
\usepackage[T2A]{fontenc}			% кодировка
\usepackage[utf8]{inputenc}			% кодировка исходного текста
\usepackage[english,russian]{babel}	% локализация и переносы
\usepackage{enumerate}
%%% Дополнительная работа с математикой
\usepackage{amsmath,amsfonts,amssymb,amsthm,mathtools,amsthm} % AMS
\usepackage{icomma} % "Умная" запятая: $0,2$ --- число, $0, 2$ --- перечисление
\usepackage{graphicx}
\usepackage{longtable}

% Двойные квадратные скобки
\usepackage{stmaryrd}
\newcommand{\llb}{\llbracket}
\newcommand{\rrb}{\rrbracket}

%% Шрифты
\usepackage{euscript}	 % Шрифт Евклид
\usepackage{mathrsfs} % Красивый матшрифтw
\ExamClass{Vitaliy Bibaev}
\ExamName{Information retrieval}
\ExamHead{\today}

%% Шрифты 
\usepackage{euscript}	 % Шрифт Евклид
\usepackage{mathrsfs} % Красивый матшрифт

%Листинг кода
%\usepackage{listings}
\usepackage{listingsutf8}
\usepackage{color}
\usepackage{bussproofs}
\renewcommand{\qedsymbol}{$\blacksquare$}
%Для листинга кода
\definecolor{mygreen}{rgb}{0,0.6,0}
\definecolor{mygray}{rgb}{0.5,0.5,0.5}
\definecolor{mymauve}{rgb}{0.63,0.082,0.082}


\lstset{
	inputencoding=utf8,
	%
	backgroundcolor=\color{white},   % choose the background color; you must add \usepackage{color} or \usepackage{xcolor}
	basicstyle=\footnotesize,        % the size of the fonts that are used for the code
	breakatwhitespace=false,         % sets if automatic breaks should only happen at whitespace
	breaklines=true,                 % sets automatic line breaking
	captionpos=b,                    % sets the caption-position to bottom
	commentstyle=\color{black},    % comment style
	deletekeywords={...},            % if you want to delete keywords from the given language
	escapeinside={\%*}{*)},          % if you want to add LaTeX within your code
	extendedchars=\true,              % lets you use non-ASCII characters; for 8-bits encodings only, does not work with UTF-8
	frame=false,                    % adds a frame around the code
	keepspaces=true,                 % keeps spaces in text, useful for keeping indentation of code (possibly needs columns=flexible)
	keywordstyle=\color{blue},       % keyword style
	morekeywords={*,...},            % if you want to add more keywords to the set
	numbers=left,                    % where to put the line-numbers; possible values are (none, left, right)
	numbersep=5pt,                   % how far the line-numbers are from the code
	numberstyle=\tiny\color{black}, % the style that is used for the line-numbers
	rulecolor=\color{white},         % if not set, the frame-color may be changed on line-breaks within not-black text (e.g. comments (green here))
	showspaces=false,                % show spaces everywhere adding particular underscores; it overrides 'showstringspaces'
	showstringspaces=false,          % underline spaces within strings only
	showtabs=false,                  % show tabs within strings adding particular underscores
	stepnumber=1,                    % the step between two line-numbers. If it's 1, each line will be numbered
	stringstyle=\color{black},     % string literal style
	tabsize=4                  % sets default tabsize to 2 spaces
	% show the filename of files included with \lstinputlisting; also try caption instead of title
}


\let\ds\displaystyle

\begin{document}

%\begin{enumerate}
	\item Рассмотрим поисковые выдачи для для трех тестовых запросов. Число на каждой позиции 
	означает оценку релевантности для соответствующих документа и запроса по следующей шкале: 
	$4 = “perfect”, 3 = “excellent”, 2 = “good”, 1 = “fair”, 0 = “bad”$. Например, число $1$ 
	на позиции $5$ в первой выдаче означает, что документ на позиции $5$ релевантен первому 
	тестовому запросу на уровне $“fair”$.
	
	\begin{tabular}{ r | c | c | c }
		rank & query 1 & query 2 & query 3 \\
		\hline
		1 & 4 & 2 & 4  \\
		2 & 1 & 2 & 4  \\
		3 & 2 & 3 & 3  \\
		4 & 3 & 4 & 3  \\
		5 & 1 & 1 & 2  \\
		6 & 3 & 3 & 1  \\
		7 & 4 & 2 & 2  \\
		8 & 0 & 4 & 0  \\
		9 & 0 & 1 & 0  \\
		10 & 1 & 1 & 0 \\
	\end{tabular}
	
	Посчитайте среднее значение $F$-метрики и $MAP$. Считайте, что документы с оценками 
	релевантности $4$ и $3$ являются релевантными, а остальные (с оценками $0–2$)	
	нерелевантными. Также считайте, что для каждого из трех тестовых запросов общее число 
	документов с оценками $4$ и $3$ во всей тестовой коллекции равно $4$. Посчитайте среднее 
	значение $DCG@3$. В данном случае, используйте пятибалльную шкалу релевантности
	
	\begin{align*}
		&F = TODO \\
		&MAP = TODO \\
		&DGC@3 = TODO
	\end{align*}
	
	\item Выпишите формулу для метрики $ERR$. Дайте определение ее составляющим. Объясните 
	смысл каждой составляющей.
	
	\begin{equation*}
		ERR = \sum\limits_{k = 1}^{N}\frac{1}{k}\cdot P(PR = \frac{1}{k}) = 
		\sum\limits_{k = 1}^{N}\frac{1}{k}\cdot \varTheta^{k - 1}\cdot R_k \prod\limits_{i = 1}^ {k - 1}(1 - R_i) 
	\end{equation*}
	Составляющие:
	\begin{enumerate}
		\item $N$  
		\item $k$
		\item $\varTheta$
		\item $R_k$
		\item $P(PR = \frac{1}{k})$
	\end{enumerate}

	\item Назовите пять онлайн метрик. Объясните, как они отражают качество поиска. Для 
	каждой метрики приведите контрпример.
	\begin{itemize}
		\item Name. Description. Example
		\item Name. Description. Example
		\item Name. Description. Example
		\item Name. Description. Example
		\item Name. Description. Example
	\end{itemize}
	
	\item $Multileaving$ [1] – это метод оценки качества поиска (онлайн), аналогичный 
	интерливингу, но работающий с $n > 2$ различными выдачами. Выпишите алгоритм работы 
	мультиливинга. Каковы преимущества и недостатки мультиливинга по сравнению с 
	интерливингом?
	
	\item Скорость сетевого соединения – $10MB$ в секунду. Каждая веб-страница занимает 
	$10KB$ и требует $500$ миллисекунд для загрузки. Сколько нужно запустить параллельных 
	потоков для обхода веба, чтобы полностью использовать возможности сетевого соединения?

	Кроме того, обходчик веба должен делать паузу в $10$ секунд между загрузками с одного и 
	того же веб-сервера. Сколько различных веб-серверов в минуту должен обходить кроулер, 
	чтобы полностью использовать возможности сетевого соединения?

	\item Рассмотрим систему поиска по электронным книгам с точки зрения контент-провайдера и 
	с точки зрения разработчиков поисковика
	
	\begin{enumerate}
		\item Вы владеете электронной библиотекой и зарабатываете на предоставлении доступа к 
		каждой конкретной книге. Вы хотите, чтобы пользователи могли находить книги из вашей 
		библиотеки с помощью сторонних поисковых систем, но чтобы доступ к книгам 
		предоставлялся только через ваш веб-сайт. Какими способами вы можете упростить работу 
		с вашими данными для разработчиков поисковиков? Как вы можете обезопасить себя от 
		несанкционированного доступа к вашим данным?
		\item Вы разрабатываете поисковик по электронным книгам
		\begin{itemize}
			\item Какими целями вы, в первую очередь, будете руководствоваться при сборе 
			данных: объем, качество, свежесть (т.е. соответствие между вашей версией 
			документа и его онлайн версией)?
			\item Какими способами вы будете собирать данные и почему?
			\item Вы можете загружать 1000 книг в день. Каким образом (приблизительно) вы 
			распределите эту квоту между загрузкой новых книг и обновлением уже загруженных и 
			почему?
		\end{itemize}
	\end{enumerate}
	
	\item  Как удаление стоп-слов и лемматизация влияют на размер индекса? Дайте	
	развернутый ответ.
	
	\item Какую минимальную информацию должен содержать индекс, чтобы обрабатывать запросы с 
	булевыми операциями? Какую информацию нужно добавить в индекс, чтобы документы можно было 
	ранжировать по их “похожести” на запрос (используйте определение “похожести”, которое вам 
	кажется разумным). Дайте развернутый ответ
\end{enumerate}

%\subsection*{Homework 2}

\begin{enumerate}
	\item \textbf{[10pt]} Рассмотренные методы исправления опечаток не работают напрямую при 
	пропуске пробела (например, \textit{informationretrieval}). Опишите, как исправлять такие 
	опечатки (не обязательно на основе рассмотренных методов).
	
	\textit{Решение.} Можно пойти двумя путями: увеличить размер индекса и время обработки запроса
	\begin{itemize}
		\item увеличение размера индекса - линейное увеличение - рассматривать как слова ещё и 
		пары слов, которые были соседними в исходном документе. Полиномиальное увеличение размера 
		индекса - рассматривать все пары/тройки/четверки/... слов, которые могут образовывать 
		слова. Наверное на практике такое сложно применить, хотя при небольшом словаре этот 
		способ может быть рабочим.  
		\item увеличение времени на обработку запроса. Пытаем разбить слово на несколько слов 
		всеми возможными способами (сначала на 2, потом на 3 и так до какого-то разумного 
		лимита). Так же можно пытаться найти соответствие префикса какому-либо слову из словаря, 
		и потом рекурсивно повторить операцию на суффиксе.
	\end{itemize}
	
	\item \textbf{[5pt]} Мы рассмотрели два типа методов для рекомендации запросов, аналогичных 
	заданному запросу:
	\begin{enumerate}
		\item Рекомендовать запросы, встречающиеся в одной сессии с заданным запросом.
		\item Рекомендовать те запросы, у которых множество кликнутых результатов сильно 	
		пересекается с аналогичным множеством для заданного запроса.
	\end{enumerate}

	Какие еще методы для рекомендации запросов вы можете предложить?
	
	\textit{Решение.} 
	\begin{itemize}
		\item Запросы других людей, за последние $N$ минут, которые похожи на текущий запрос 
		(например, пересекаются по некоторым словам). Пример: $q=$\textit{Презентация ...}, 
		возможная подсказка \textit{"Презентация apple"}. Т.к. она (условно) была вчера вечером и 
		уже многие сегодня весь день хотели найти об этом информацию.
		\item Запросы других людей, которые были в это же время суток/в тот день недели/в таких 
		же числах месяца/в то же время года, которые похожи на текущий запрос (например, 
		пересекаются по некоторым словам). Пример: пятница, полдень $q=$ \textit{"погода на "}. 
		Логичнее дополнить как \textit{"на выходные"}. То же самое работает и для праздников.
		\item Запросы которые уже встречались от этого же пользователя и похож на текущий. Часто 
		бывает нужно повторить поиск, но не всегда пользователь запоминает абсолютно точно текст 
		запроса, но это может влиять на выдачу.
		\item Если известна геопозиция пользователя(либо история запросов с какими-либо 
		гео-данными), то можно дополнить запрос соответствующей информацией.
		
		Пример: $q=$\textit{Погода в}. Возможная рекомендация \textit{Погода в Турции}, если он уже 
		упоминал Турцию, когда искал билеты и отель.
	\end{itemize}
	\item \textbf{[5pt]} Вы планируете использовать следующие методы поиска: модель векторного 
	пространства с весами TF-IDF, BM25, языковую модель. Какую минимальную информацию должен 
	содержать индекс, чтобы поддерживать эффективное использование этих методов? Какую информацию 
	нет смысла хранить в индексе? Как ее нужно хранить? Дайте развернутый ответ.
	
	\textit{Решение.} TODO: Add parameters for the language model 
	
	Рассмотрим используемые параметры в этих методах.
	\begin{itemize}
		\item $tf(t, d)$ - частота слова внутри документа (term frequency)
		\item $df(t)$ - частота слова среди всех документов
		\item $ifd(t)$ - обратная частота слова
		\item $dl(d)$ - длина документа
		\item $dl_{ave}$ - средняя длина документа
		\item $N$ - количество всех документов
	\end{itemize}
	
	Несложно догадаться, что индекс должен содержать хотя бы отображение из слова в список 
	документов, в которых оно встречается, а так же количество вхождений этого слова в 
	соответствующий документ. Назовём этот индекс $I$. Это позволит вычислить $tf(t, d)$ - найти 
	$I(t)$ документ $d$ и взять количество вхождений. $df(t)$ - это просто длина списка $I(t)$. 
	
	Для вычисления оставшихся характеристик, можно включить в индекс отображение $D$ - документ 
	$\rightarrow$ его длина. Ещё нужно хранить суммарную длину всех документов - $L$. Тогда:
	\begin{align*}
		dl(d) &= D(d) \\
		N &= |D|\\
		dl_{ave} &=\frac{L}{N}
	\end{align*}
	Теперь, зная $N$, можно вычислить $idf(t)$ по понравившейся формуле.
	
	Любая дополнительная информация будет избыточной. Например, лишено смысла хранить $dl_{ave}$ в 
	явном виде, т.к. при добавлений документа его будет сложно обновить.
	
	Первый индекс ($I$) можно хранить как хеш-таблицу, второй индекс ($D$) как обычный 
	массив(файл) с прямой адресацией, где индекс - это идентификатор документа (тогда id 
	документов нужно выбрать соответствующим образом - как целые неотрицательные числа)
	
	\item \textbf{[10pt]} Отранжируйте документы из таблицы 1 по запросу "car insurance" с 
	использованием модели векторного пространства и весов TF-IDF.
	
	\begin{tabular}{c | c | c c c}
		\multirow{2}{*}{Слово} & \multirow{2}{*}{idf} & \multicolumn{3}{c}{tf} \\ \cline{3-5}
		& & Документ 1 & Документ 2 & Документ 3 \\ \hline	
		car & 1.65 & 27 & 4 & 24 \\
		auto & 2.08 & 3 & 33 & 0 \\
		insurance & 1.62 & 0 & 33 & 29 \\
		best & 1.60 & 14 & 0 & 17 \\
	\end{tabular}
	
	\textit{Решение.} Определим веса TF-IDF для каждого из документов ($d_i = v (document \ i)$).
	\begin{align*}
		d_1 &= \{27 \cdot 1.65, 3 \cdot 2.08, 0 \cdot 1.62, 14 \cdot 1.60\} = \{44.55, 6.24, 
		0.00, 22.40\} \\
		d_2 &= \{4 \cdot 1.65, 33 \cdot 2.08, 33 \cdot 1.62, 0 \cdot 1.60\} = \{6.60, 68.64, 
		53.46, 0.00\} \\
		d_3 &= \{24 \cdot 1.65, 0 \cdot 2.08, 29 \cdot 1.62, 17 \cdot 1.60\} = \{39.60, 0.00, 
		46.98, 27.20\}
	\end{align*}
	
	А так же вектор для запроса: 
	\begin{align*}
		v(q) = \{1.65, 0.00, 1.62, 0.00\}
	\end{align*}
	
	Осталось вычислить значения $sim(d_i, v(q))$ и ранжировать запросы.
	\begin{equation*}
		sim (d_i, v(q)) = \frac{d_i \cdot v(q)}{||d_i||\cdot ||v(q)||}
	\end{equation*}
	\begin{align*}
		sim(d_1, v(q)) &= \frac{44.55 \cdot 1.65}{\sqrt{44.55^2 + 6.24^2 + 22.40^2} \cdot 
				\sqrt{1.65^2 + 1.62^2}} = 0.63 \\
		sim(d_2, v(q)) &= \frac{6.60 \cdot 1.65 + 53.46\cdot 1.62}{\sqrt{6.60^2 + 68.64^2 + 
				53.46^2} \cdot \sqrt{1.65^2 + 1.62^2}} = 0.48 \\
		sim(d_3, v(q)) &= \frac{39.60 \cdot 1.65 + 46.98\cdot 1.62}{\sqrt{39.60^2 + 46.98^2 + 
				27.20^2} \cdot \sqrt{1.65^2 + 1.62^2}} = 0.91 \\
	\end{align*}
	Таким образом, ранжирование будет следующим:
	\begin{enumerate}
		\item[1:] Документ 3
		\item[2:] Документ 1
		\item[3:] Документ 2
	\end{enumerate}
	
	\item \textbf{[5pt]} Рассмотрим следующий запрос и три результата.
	\begin{enumerate}
		\item[Q] information retrieval course
		\item[D1] Information Retrieval and Web Search
		\item[D2] Introduction to Information Retrieval
		\item[D3] Text Retrieval and Search Engines
	\end{enumerate}
	
	Результаты $1$ и $3$ – это страницы соответствующих курсов, поэтому пользователь пометил их 
	как релевантные. Результат $2$ – это страница с книгой, поэтому пользователь пометил его как 
	нерелевантный.
	
	Примените алгоритм Роккио и выпишите вектор запроса после учета обратной связи по 
	релевантности. Элементы вектора перечислите в алфавитном порядке. Считайте, что компоненты 
	векторов содержат только частоты слов (без обратной документной частоты и нормировки). 
	Параметры алгоритма Роккио: $\alpha = 1, \beta = 0.75, \gamma = 0.25.$

	\textit{Решение. } Все что нам нужно-воспользоваться формулой:
	\begin{align*}
		q_{opt} = \alpha v(q) + \beta \mu(D_r) - \gamma \mu(D_{nr})
	\end{align*}
	
	Выпишем таблицу с частотами слов в документах и запросе
	
	\begin{tabular}{l | c c c | c}
						& \textbf{D1} & D2 & \textbf{D3} & Q \\ \hline
		and 			& 1 & 0 & 1 & 0 \\
		course 			& 0 & 0 & 0 & 1 \\
		engines		 	& 0 & 0 & 1 & 0\\
		information		& 1 & 1 & 0 & 1 \\
		introduction 	& 0 & 1 & 0 & 0\\\hline
		retrieval 		& 1 & 1 & 1 & 1 \\
		search 			& 1 & 0 & 1 & 0 \\
		text 			& 0 & 0 & 1 & 0\\ 
		to			 	& 0 & 1 & 0 & 0\\
		web 			& 1 & 0 & 0 & 0 \\
	\end{tabular}
	
	Таким образом, получили следующие вектора:
	\begin{align*}
		v(D1) &= \{1, 0, 0, 1, 0, 1, 1, 0, 0, 1\} \\
		v(D2) &= \{0, 0, 0, 1, 1, 1, 0, 0, 1, 0\} \\
		v(D3) &= \{1, 0, 1, 0, 0, 1, 1, 1, 0, 0\} \\
		v(q) &= \{0, 1, 0, 1, 0, 1, 0, 0, 0, 0\} \\
	\end{align*}
	
	Для вычисления $q_{opt}$ потребуются $D_{r}, D_{nr}$ - множество векторов для релевантных и 
	нерелевантных запросов.
	\begin{align*}
		D_r = \{v(D1), v(D3)\}
		D_{nr} = \{v(D2)\}
	\end{align*}
	
	Вычислим $\mu(D_r)$:
	\begin{align*}
		\mu(D_r) = \frac{1}{2} (v(D1) + v(D3)) = \frac{1}{2}\cdot (\{1, 0, 0, 1, 0, 1, 1, 0, 
		0, 1\} + \{1, 0, 1, 0, 0, 1, 1, 1, 0, 0\}) = \\
		= \{1, 0, 0.5, 0.5, 0, 1, 1, 0.5, 0, 0.5\}
	\end{align*}
	
	И $\mu(D_{nr})$:
	\begin{equation*}
	\mu(D_{nr}) = v(D2) = \{0, 0, 0, 1, 1, 1, 0, 0, 1, 0\}
	\end{equation*}
	
	Осталось вычислить результат
	
	\begin{align*}
		q_{opt} = \alpha v(q) + \beta \mu(D_r) - \gamma \mu(D_{nr}) = 1 \cdot q + 0.75 \cdot 
		\mu(D_r) - 0.25 \cdot \mu(D_{nr}) = 
		= \{0, 1, 0, 1, 0, 1, 0, 0, 0, 0\} + \\ + 0.75 \cdot \{1, 0, 0.5, 0.5, 0, 1, 1, 0.5, 
		0, 0.5\} - 0.25 \cdot \{0, 0, 0, 1, 1, 1, 0, 0, 1, 0\} = \\ = (0.75, 1, 0.375, 
		1.125, -0.25, 1.5, 0.75, 0.375, -0.25, 0.375)
	\end{align*}

	\item \textbf{[10pt]} Выпишите формулу \textit{BM25} для длинных запросов. Опишите ее 
	составляющие. Каким образом каждая составляющая влияет на ранжирование (т.е. что происходит	с 
	ранжированием результатов при изменении каждой из составляющих)?

	\textit{Решение. } Формула для $BM25$ выглядит так:
	\begin{align*}
	\sum\limits_{t \in q} \log \left[ \frac{N}{df(t)} \right] \cdot \frac{(k_1 + 1) \cdot 
		tf(t, d)}{k_1\cdot \left[ (1 - b) + b \frac{dl(d)}{dl_{ave}} \right] + tf(t, d)} 
	\cdot \frac{(k_3 + 1) \cdot tf(t, q)}{k_3 + tf(t, q)}
	\end{align*}
	
	Составляющие:
	\begin{itemize}
		\item $q$ - запрос
		\item $N$ - количество документов в коллекции
		\item $tf(t, d)$ - количество вхождений терма $t$ в документ $d$
		\item $df(t)$ - количество документов в коллекции, в которые входит терм $t$
		\item $dl(d)$ - длина документа $d$
		\item $dl_{ave}$ - средняя длина документа во всей коллекции
		\item $tf(t, q)$ - количество вхождений терма $t$ в запрос $q$
		\item $k_1, b, k_3$ - параметры метода. 
	\end{itemize}
	
	Влияние
	\begin{itemize}
		\item $N, df(t)$ - по-отдельности про них мало что можно сказать, лучше рассмотреть их 
		отношение
		\item  $\dfrac{N}{df(t)}$ - не зависит от документа, устанавливает "вес" для терма $t$ - 
		чем больше, тем лучше терм (т.е лучше определяет требуемую выдачу). Иными словами, чем 
		больше значение - тем меньше документов, которые содержат этот терм. 
		\item $tf(t, d)$ - количество вхождений терма $t$ в документ $d$. Чем чаще встречается 
		терм в документе, тем этого документ лучше подходит.
		\item $dl(d), dl_{ave}$ - абсолютные величины так же мало о чем говорят, поэтому 
		рассмотрим относительную величину - отношение длины документа к средней длине
		\item $\dfrac{dl(d)}{dl_{ave}}$ - чем больше, тем ниже ранг документа. Если документ 
		очень длинный, то в нем может быть много слов из запроса, но не потому что он подходит, а 
		просто потому что он слишком длинный и содержит много слов. 
		\item $tf(t, q)$ - чем чаще слово встречается в тексте запроса тем сильнее пользователь 
		его ищет. Это тоже "вес" для терма.
		\item $k_1$ - повышаем вклад для значения $tf(t, d)$, то есть выше будут документы, 
		которые чаще содержат терм $t$
		\item $b$ - выглядит как некоторый "параметр релаксации" для отношения 
		$\frac{dl(d)}{dl_{ave}}$. Ближе к $1$ - хотим учитывать среднюю длину документа, ближе к 
		$0$ - не хотим. Средние значения учесть, но с каким-то дополнительным весом.
		\item $k_3$ повышаем вклад для значения $tf(t, q)$, термы от, которые часто встречаются в 
		теле запроса будут давать больший вклад в скор для документов
	\end{itemize}

	\item \textbf{[10pt]} Пусть бинарная случайная величина $X_t$ – это индикатор того, что слово 
	$t$	встречается в документе (т.е. $X_t = 1$, если слово $t$ есть в документе, и $X_t = 0$, 
	если слова $t$ нет в документе). $P_t = P(X_t = 1 \big| d)$ – это вероятность того, что слово 
	$t$ встречается в документе $d$. 
	
	Примените метод максимального правдоподобия \textit{(MLE)} для формального вычисления $P_t$ и 
	покажите, что $P_t = \dfrac{tf(t,d)} {dl(d)}$, где 
	\begin{itemize}
		\item $tf(t, d)$ – это частота слова $t$ в документе $d$, 
		\item $dl(d)$ – это длина документа $d$.
	\end{itemize}
	
	\textit{Решение. } В нашем случае случайная величина - дискретная с двумя значениями - $0, 
	1$. Значит, функция плотности для неё - $p_0 = P(X_t = 0 | d) = (1 - \theta)$, $p_1 = 
	P(X_t = 1 | d) = \theta$. Пусть у нас есть $n$ элементов из распределения нашей случайной 
	величины, выпишем функцию правдоподобия в общем виде:
	\begin{equation*}
		\mathcal{L}(x_i; \theta) = \theta^k(1-\theta^{n-k})
	\end{equation*}
	
	где $k$ - количество случайных $x_i$, которые совпали с $t$. Найдем максимум функции правдоподобия. 
	\begin{equation*}
		\mathcal{L}'_\theta = k\theta^{k - 1}(1 - \theta^{n - k}) - (n - k)\theta^k (1 - \theta^{n - k - 1})
	\end{equation*}
	
	Найдём точку экстремума при которой $\mathcal{L}'_\theta = 0$
	\begin{align*}
		&k\theta^{k - 1}(1 - \theta^{n - k}) - (n - k)\theta^k (1 - \theta^{n - k - 1}) &= 0 \\
		&\theta^{k - 1}(1 - \theta^{n - k - 1})\left[ k (1 - \theta) - (n - k)\theta \right] &= 0 \\
		 &k (1 - \theta) - (n - k)\theta &= 0 \\
		 &k - k \theta - n \theta + k\theta &= 0 \\
		 &k - n\theta &= 0 \\
		 \\
		 \Rightarrow \theta &= \dfrac{k}{n}
	\end{align*}
	
	Осталось убедиться, что это точка максимума - $\mathcal{L}(x_i; 0) =\mathcal{L}(x_i; 1) = 0$. 
	Далее, мы могли бы определять знаки производной слева/справа от найденной точки экстремума,но 
	можем ограничиться следующим замечанием: на отрезке $[0, 1]$ это единственный экстремум, 
	значение в нём положительное, поэтому найденная точка ($\theta = \frac{k}{n}$) максимизирует 
	функцию правдоподобия.
	
	Выбрав $n = dl(d)$, т.е. включив в выборку все слова документа, получим требуемое утверждение.
	
	\item \textbf{[5pt]} Рассмотрим коллекцию из двух документов.
	\begin{enumerate}
		\item[D1] A language model is a probability distribution over words or sequences of words.
		\item[D2] A language model is used in many natural language processing applications.
	\end{enumerate}

	Выпишите сглаженную униграмную языковую модель для каждого документа. Используйте сглаживание 
	\textit{Jelinek-Mercer} с параметром $\lambda = 0.5$. Отранжируйте эти документы по запросу 
	\textit{"many words"}.
	
	\textit{Решение.} Во первых выпишем формулу для того, чтобы оценить ранг документа:
	\begin{equation*}
		P(q \ |\ M_d) = \prod\limits_{t\in q}P_s(t\ |\ M_d)
	\end{equation*}
	
	В нашем случае множители правой части определяются выражением:
	\begin{equation*}
		P_s(t\ |\ M_d) = \lambda P(t\ |\ M_d) + (1 - \lambda) P(t\ |\ M_c)
	\end{equation*}
	
	где $P(t\ |\ M_d) = \dfrac{tf(t, d)}{dl(d)}$, а $P(t\ |\ M_c) = \dfrac{cf(t)}{cl}$
	
	Для того, чтобы применить это сглаживание нам нужно значить размер коллекции термов($cl$), а так же частоты термов запроса в этой коллекции$cf(t)$. В нашем случае, состав коллекции определяется всеми словами $D1 \cup D2$, причем, с повторениями.
	
	\begin{align*}
		cl = len(D1) + len(D2) = 13 + 11 = 24
		cf(many) = 1
		cf(words) = 2
	\end{align*}
	
	Осталось всё подставить в формулу и вычислить:
	\begin{align*}
		P(q\ |\ M_{d1}) = \left[\lambda P(many\ |\ M_{d1}) + (1 - \lambda) P(many\ |\ M_c)\right] \cdot 
		\left[\lambda P(words\ |\ M_{d1}) + (1 - \lambda) P(words\ |\ M_c)\right] = \\
		= \left[ \lambda \frac{0}{13} + (1 - \lambda) \frac{1}{24} \right] \cdot \left[ \lambda 
		\frac{2}{13} + (1 - \lambda) \frac{2}{24} \right] = 0.5 \frac{1}{24} \cdot \left[ 0.5 \frac{2}{13} + 0.5 \frac{2}{24} \right] = 0.021
	\end{align*}
	
	\begin{align*}
	P(q\ |\ M_{d2}) = \left[\lambda P(many\ |\ M_{d2}) + (1 - \lambda) P(many\ |\ M_c)\right] \cdot 
	\left[\lambda P(words\ |\ M_{d2}) + (1 - \lambda) P(words\ |\ M_c)\right] = \\
	= \left[ \lambda \frac{1}{11} + (1 - \lambda) \frac{1}{24} \right] \cdot \left[ \lambda 
	\frac{0}{11} + (1 - \lambda) \frac{2}{24} \right] = \left[ 0.5 \frac{1}{11} + 0.5 \frac{1}{24} \right] \cdot 0.5 \frac{2}{24} = 0.020
	\end{align*}
	
	Таким образом ранг первого документа будет выше, чем ранг второго
	
\end{enumerate}

\subsection*{Homework 3}

\begin{enumerate}
	\item \textbf{[10pt]} Пусть следующая матрица является матрицей смежности ``термин-документ'', 
	описывающей некую коллекцию:
	\begin{equation*}
	C = 
		\begin{bmatrix}
			1 & 1 \\
			0 & 1 \\
			1 & 0
		\end{bmatrix}
	\end{equation*}
	\begin{itemize}
		\item 	Вычислите матрицу совместной встречаемости $CC^T$ . Что собой представляют 
		диагональные элементы этой матрицы?
		\item Убедитесь, что сингулярное разложение матрицы C выглядит следующим образом:
		\begin{equation*}
			\mathcal{U} = 
			\begin{bmatrix}
			-0.816 & 0.000 \\
			-0.408 & 0.707 \\
			-0.408 & 0.707
			\end{bmatrix}, \
			\Sigma = 
			\begin{bmatrix}
			1.732 & 0.000 \\
			0.000 & 1.000
			\end{bmatrix}, and \
			V^T = 
			\begin{bmatrix}
			-0.707 & -0.707 \\
			0.707 & - 0.707
			\end{bmatrix}
		\end{equation*}
		\item Что представляют собой элементы матрицы $C^TC$?
	\end{itemize}
	\textit{Решение.} 
	
	\begin{equation*}
	CC^T = 
	\begin{bmatrix}
		-0.816 & 0.000 \\
		-0.408 & 0.707 \\
		-0.408 & 0.707
	\end{bmatrix}
	\end{equation*}
	
	Диагональные элементы - скалярное произведение строки на себя. В случае матрицы 
	"термин-документ" это соответствует количеству документов, в которых входит данный термин.
	
	Для того, чтобы убедиться, что матрицы $\mathcal{U}, \Sigma, V^T$ являются синглядрным 
	разложением матрицы $C$ достаточно найти их произведение $\mathcal{U} \Sigma V^T$.
	
	 \begin{align*}
	 \mathcal{U} \Sigma V^T = 
	 \begin{bmatrix}
		 -0.816 & 0.000 \\
		 -0.408 & 0.707 \\
		 -0.408 & 0.707
	 \end{bmatrix}
	 \cdot 
	 \begin{bmatrix}
		 1.732 & 0.000 \\
		 0.000 & 1.000
	 \end{bmatrix} \cdot 
	 \begin{bmatrix}
		 -0.707 & -0.707 \\
		 0.707 & - 0.707
	 \end{bmatrix} = \\
	 \begin{bmatrix}
		 -1.41331 & 0 \\
		 -0.706656 & 0.707 \\
		 -0.706656 & 0.707 
	 \end{bmatrix} \cdot
	 \begin{bmatrix}
		 -0.707 & -0.707 \\
		 0.707 & - 0.707
	 \end{bmatrix}
	 = 
	 \begin{bmatrix}
		 1 & 1 \\
		 0 & 1 \\
		 1 & 0
	 \end{bmatrix} = C,
	 \end{align*}
	 
	 Вычислим матрицу $C^TC$. 
	 
	 \begin{equation*}
	 C^TC = 
	 \begin{bmatrix}
		 2 & 1 \\
		 1 & 2
	 \end{bmatrix}
	 \end{equation*}
	 
	 Заметим, что элементы этой матрицы - это скалярные произведения строк матрицы $C$, то есть 
	 это скалярные произведения векторов термин - документы. Чем больше это произведения, тем 
	 ближе эти вектора, некоторые модели выводят из этого, что слова и по смыслу тоже близки.
	
	\item \textbf{[10pt]} Для чего используются распределения Дирихле $Dir(\alpha)$ и $Dir(\beta)$ 
	в тематических моделях? Что контролируют параметры $\alpha$ и $\beta$ (нужно иметь в виду, что 
	$\alpha$ и $\beta$ - это наборы/векторы параметров)? Какие значения этих параметром имеет 
	смысл использовать и почему?
	
	\textit{Решение.}
	
	Введем обозначения: $k$ - количество тем, $n$ - размер словаря, $m$ - количество документов.
	
	\begin{itemize}
		\item $Dir(\alpha)$ - позволяет получить распределение тем по документу. $\alpha$ - вектор 
		из $k$ положительных вещественных значений. $i$ - ая компонента задает вес темы под 
		номером $i$ в документе. Обычно компоненты вектора - одинаковые (если нет априорного 
		знания о распределении тем) значения меньше 1 - порядка - $\frac{1}{k}$. Не стоит брать 
		слишком большие значения $\alpha_i$, т.к. в этом случае тем в одном документе может 
		оказаться слишком мало, и наоборот, слишком много, если взять очень маленькие значения для 
		$\alpha_i$. 
		
		\item $Dir(\beta)$ - позволяет получить распределение слов по темам. $\beta$ - вектор из 
		$n$ положительных вещественных значений. $i$ - ая компонента задает вес слова под номером 
		$i$ в теме. Из тех же соображений стоит взять значения сильно меньше 1 - порядка 
		$\frac{1}{n}$, чтобы в одну тему попало несколько слов, а не одно/сразу все, может 
		получиться в крайних случаях.
	\end{itemize}
	
	\item \textbf{[5pt]} Пользователь в дополнение к кликам по гиперссылкам на странице, которую 
	он в данный момент просматривает, может перейти по кнопке "назад" и вернуться на предыдущую 
	страницу. Можно ли эту ситуацию смоделировать с помощью марковской цепи и как? Как 
	смоделировать повторяющиеся щелчки на кнопке "назад"?
	
	\textit{Решение.}
	
	Да, это возможно, но нужно с каждой вершиной связать стек, и добавлять в него предыдущую 
	вершину пути, при попадании в вершину. А при переходе из вершины нужно добавить вариант - 
	перейти "назад", с фиксированной вероятностью (либо более сложной). После перехода, элемент со 
	стека нужно снять. Использование стека позволит так же выполнять повторяющиеся щелчки, даже 
	когда движение "назад" образует циклы. 
	
	Осталось решить проблему, что цепи Маркова не знают ничего про стек. Но зато они могут 
	содержать бесконечное (счетное число вершин). Элементы стека - это номера вершин исходной цепи 
	Маркова, значит, стек имеет счетное число состояний (вообще говоря, даже если множество 
	состояний исходной цепи тоже счетное, но количество сайтов - конечное число). Теперь вершину и 
	состояние стека можно вынести в отдельную вершину. Переходы "назад" и "вперед" просто переходят 
	в вершины с соответствующими состояниями стека.

	\item \textbf{[10pt]} Объясните основную идею метода $Ranking \ SVM$ – одного из первых 
	методов	\textit{pairwise learning to rank [1]}. В частности, опишите, что оптимизирует этот 
	метод, какие тренировочные данные он использует и как получить эти данные.
	
	\textit{Решение.}
	
	Цель - имея только данные логов получить функцию ранжирования. Идея - документы, которые внизу выдачи, но при этом были выбраны пользователем релевантнее запросу $q$, тех, которые были выше, но пользователь их пропустил.
	
	Данные - тройки $(q, r, c)$ - где $q$ - запрос, $r$ - ранжирование, $c$ - набор кликов. Чтобы 
	их собрать нужно модифицировать результат поисковой выдачи и воспользоваться прокси сервером 
	(данные собираются неявно для пользователя). Алгоритм следующий - прежде чем показать 
	пользователю выдачу, она проходит через прокси, запросу присваивается уникальный $id$, а 
	ссылки на документы заменяются на ссылки на этот же прокси (с сохранением исходного адреса 
	документа) и в таком виде передается пользователю. После клика по документу запрос приходит на 
	прокси сервер, где можно получить данные о клике для запроса с таким $id$ и перенаправить 
	пользователя на "настоящий" документ.
	
	После того, как данные собраны из них можно извлечь всё предпочтения пользователей - некоторое 
	отношение частичного порядка на множестве документов. Считается, что документ $d_1$ 
	предпочтительнее $d_2$, если $d_1$ оказался ниже в выдаче, при этом по $d_1$ был клик, а по 
	$d_2$ нет. Пару $(d_1, d_2)$ будет называть предпочтение.
	
	Ранжирование считается хорошим, если нарушает как можно меньше предпочтений. Таким образом 
	тренировочная выборка это набор пар (запрос, набор предпочтений). Искомая функция - функция 
	ранжирования $f(q)$. Оптимизируемая величина - средняя доля нарушенных предпочтений среди 
	элементов обучающей выборки (вообще, это довольно грубое утверждение, оптимизируется Kendall's 
	$\tau$)
	
	\item \textbf{[5pt]} Объясните методы нормировки \textit{Z-Score} и \textit{Sum} с точки 
	зрения предполагаемого статистического распределения нормируемых данных. Т.е. какие 
	распределения предполагают эти методы и что они делают с предполагаемыми распределениями?
	
	\textit{Решение.}
	\begin{itemize}
		\item \textit{Z-Score} - основывается на предположении, что данные распределены нормально. Переводит результаты к стандартному нормальному распределению - с нулевым матожиданием и единичной дисперсией.
		\begin{align*}
			s_{norm} &= \frac{s - \mu}{\sigma}, where\\
			\mu &= \frac{1}{n} \sum\limits_{i = 1}^n s_n\\
			\sigma &= \sqrt{\frac{1}{n} \sum\limits_{i = 1}^n (s_i - \mu)}
		\end{align*}
		
		\item \textit{Sum} - основывается на предположении, что данные распределены экспоненциально.
		\begin{equation*}
			s_{norm} = \frac{s'}{\sum\limits_{i = 1}^{n}s'_i}, where \ s' = s - min, 
		\end{equation*}
	\end{itemize}
	
	
	\item \textbf{[5pt]} Рассмотрим задачу поиска экспертов \textit{(expert finding)} в той или 
	иной области знаний, где область знаний – это запрос, а эксперты – это сущности, которые нужно 
	отранжировать по запросу.
		
	Например, по запросу \textit{``information retrieval''} из множества всех известных 
	экспертов	нужно выбрать (и отранжировать) ``Bruce Croft'', ``Christopher Manning'', 
	``Maarten de Rijke'', ``Ilya Markov'' и т.д.
	
	Каждый эксперт является автором некоторого количества документов (научные статьи, патенты, 
	письма и т.д.). Какие из методов, рассмотренных в лекциях, можно использовать для задачи 
	поиска экспертов и каким образом?
	
	\textit{Решение.}
	\item \textbf{[5pt]} Выведите формулы для подсчета полной и условной вероятности клика для	
	позиционной кликовой модели \textit{(PBM)}. Как они соотносятся друг с другом и почему?
	
	\textit{Решение.}
	\begin{align*}
		P(C_u &|E_{r_u} = 1 ) = \alpha_{uq}\\
		P(C_u &|A_u = 1) = \gamma_{r_u}\\
		P(C_u &|E_{r_u} = 0 , A_u = 0) = 0\\
		P(C_u &|E_{r_u} = 0 , A_u = 1) = 0\\
		P(C_u &|E_{r_u} = 1 , A_u = 0) = 0\\
		P(C_u &|E_{r_u} = 1 , A_u = 1) = 1 \\
		P(C_u &= 1) = P(E_{r_u} = 1) \cdot P(A_u = 1) = \gamma_{r_u} \cdot \alpha_{uq}
	\end{align*}
	
	\item \textbf{[5pt]} Рассмотрим кликовую модель \textit{dynamic Bayesian network (DBN)}. Она 
	включает три типа параметров:
	
	\begin{itemize}
		\item Аттрактивность $\alpha_{uq}$ – пользователю понравился сниппет.
		\item Удовлетворенность $\sigma_{uq}$ – пользователю понравился сам документ.
		\item Вероятность продолжить чтение сниппетов $\gamma$ в случае неудовлетворенности.
	\end{itemize}
	
	Параметр $\gamma$ обычно близок к 1, поэтому будет считать, что $\gamma = 1$. Как в этом 
	случае подсчитать значения параметров $\alpha_{uq}$ и $\sigma_{uq}$ на основании некоторого 
	имеющегося лога кликов?
	
	\textit{Решение.}
\end{enumerate}

\end{document}
